% !TEX root = main.tex

\documentclass[11pt]{article}

\usepackage{amsmath,amssymb,amsfonts}
\usepackage{graphicx}
\usepackage{tikz}
\usepackage{multirow}
\graphicspath{{media/}}

\setlength{\topmargin}{-.5in} \setlength{\textheight}{9.25in}
\setlength{\oddsidemargin}{0in} \setlength{\textwidth}{6.8in}


\begin{document}
% !TEX root = main.tex

\begin{center}
    \newcommand{\HRule}{\rule{\linewidth}{0.5mm}}
    \begin{minipage}{0.48\textwidth} 
        \begin{flushleft}
            \includegraphics[scale = 0.08]{logo_unam.png}
        \end{flushleft}
    \end{minipage}
    \begin{minipage}{0.48\textwidth} 
        \begin{flushright}
            \includegraphics[scale =0.22]{logo_ciencias.png}
        \end{flushright}
    \end{minipage}
    \vspace*{-1.5cm}						
    \textsc{\huge Nacional Autónoma de México \\ \vspace{-4px} Universidad }\\[2cm]	
    \textsc{\LARGE Facultad de Ciencias}\\[1.5cm]
    \vspace*{1cm}					
        \HRule \\[0.4cm]							
            { \huge \bfseries Tarea-Examen 1}\\[0.4cm]	
        \HRule \\[1.5cm]						    
    \begin{minipage}{0.52\textwidth}													
        \begin{flushleft} \large						\small				\vspace{-0.6cm}	
            \vspace{-0.6cm}	
            \emph{Integrantes:}\\
           Yonathan Berith Jaramillo Ramírez. 419004640 \\
           Dalia Viridiana Martinez Retiz. 309213293 \\
            \vspace*{2cm}
        \end{flushleft}																		
    \end{minipage}		
    \begin{minipage}{0.46\textwidth}		
        \vspace{-0.6cm}											
        \begin{flushright} \large						\small										
            \emph{Profesor:} 
            Ruth Selene Fuentes García \\
           \emph{Ayudante:} 
            		Carlos Alberto Arriaga Solórzano \\
            
        \end{flushright}																	
    \end{minipage}	
    \vspace*{1cm}
    % \center{\textbf{\Large Ciencias de la computación}}\\
    \vspace{2cm}
    \begin{center}						
    \includegraphics[scale=0.13]{Estadistica-1.jpg}
    \end{center}  	
        \begin{center}						
        {\large 13 Septiembre, 2023}
    \end{center}  
\end{center}	
\textbf{}
\Large

\noindent{\bf Estadística 1 \hfill Tarea 1}

\medskip\hrule
\vspace{1cm}

\begin{enumerate}
    \item Resume la informacion de los siguientes datos como te parezca mas conveniente.
    \item Las calificaciones de un grupo de 18 estudiantes en una escala de 100 puntos: 31, 35, 37, 40, 40, 51, 54, 55,
57, 58, 60, 60, 62, 62 65, 67, 75, 89.
\item Los dias laborables que han perdido 20 trabajadores: 0, 0, 0, 0, 0, 0, 0, 1, 1, 1, 1, 2, 3, 3, 4, 4, 5, 5, 8, 45

\item La estatura de 20 mujeres en metros que se estudia con relacion a una condicion medica: 1.52, 1.60, 1.57, 1.52, 1.60, 1.75, 1.73, 1.63, 1.55, 1.63, 1.65, 1.55, 1.65, 1.60, 1.68 , 2.50, 1.52, 1.65, 1.60, 1.65

\item Los datos contenidos en la base mtcars en R 
\begin{center}
    Base\_descriptivo.csv
\end{center}
y para la variables post carga 2 hrs y colesterol total, ¿considera usted que su comportamiento es similar
en toda la muestra?

\item Suponga que $X_{1}$, . . . , $X_{n}$ son variables aleatorias iid distribuidas Binomial(3,p). Obtenga la distribucion para Y =
$\sum_{i=1}^{n} X_{i}$ Considere Z = Y /(3n), ¿ Cuales son los posibles valores que toma Z, ¿ Como se distribuye Z? Note: que Z es una variable aleatoria discreta.

\item Considere dos variables aleatorias independientes \textit{u} $\sim$ \textit{X}$_{p}^{2}$ y \textit{u} $\sim$ \textit{X}$_{q}^{2}$, muestre que \textit{f} = $\frac{u/p}{v/q}$ se distribuye $F_{p,q}$

\end{enumerate}

\end{document}
